\documentclass[12pt, letterpaper]{article}
\usepackage[utf8]{inputenc}
\usepackage{geometry}
\geometry{
    margin=1in
}

\title{%
    PADLOCK \\
    \large Protocol Allowing for Decentralized Locking/Ownership of C-hashes to
    K-public keys
}
\author{Ellis Frank}

\begin{document}

\maketitle

\begin{abstract}
    PADLOCK is a blockchain based protocol that allows for the creation of a
    scalable and private cryptocurrency. It achieves this by only storing a hash
    of the input and the output, only a hash of the signature, and then a 8 byte
    anti-spam proof of work. These are the only details that need to be publicly
    disclosed on the blockchain in order for a transaction recipient to be able
    to verify that the coins they have received were not double spent. This
    results in an incredibly small transaction size of only 56 bytes, allowing
    for around 4x more throughput than bitcoin.
\end{abstract}


\section{Introduction}
PADLOCK is a blockchain based system that allows the ownership and transfer of
hashes by key pairs. The purpose of this hash ownership is to provide the base
for creating a scalable and private currency that runs on top of the PADLOCK
blockchain. This paper assumes a general understanding of how traditional
cryptocurrencies work.

\section{How Does PADLOCK Work?}
PADLOCK uses digital signatures to facilitate the ownership of hashes. However,
it does it differently to how most systems work. Only a hash of the signature is
stored on-chain, and the public key is left off-chain. Alongside the signature
hash is an input hash, an output hash, and a proof of work. These four together,
make up an \textit{entry}.

\begin{verbatim}
\\ Example written in Rust
type Hash = [u8; 16];

struct Entry {
    input: Hash,
    output: Hash,
    signature_hash: Hash,
}
\end{verbatim}

Anyone can submit any entry to the blockchain. They do not have to provide any
sort of signature, just an object of three 16 byte arrays. This provides a base
for creating a scalable, private, and non-double-spendable cryptocurrency.


\section{Creating Coins on PADLOCK}

\subsection{What is a Coin}
It's important to make a distinction when talking about coins in this context.
Coins, in this context, do not refer to a unit of currency, but rather a chain
of digital signatures. When you transfer a coin, you are appending a signature
to this history. Each transaction has an \textit{input} and \textit{output}. The
input is the hash of ttransaction that the sender the coin, and the output is the new
owner(s) of the coin. The input is represented as the hash of the previous
transaction.

\begin{verbatim}
    \\ Example written in Rust
    struct Coin {
        transactions: Vec<Transaction>,
    }
    
    struct Transaction {
        input_hash: Hash,
        output: Vec<PublicKey>,
        hash: Hash,
        signature: [u8; 64],
    }
\end{verbatim}

This coin can be split into multiple coins as well, specifying more than one
public key as the new owner, each representing a fraction of the value of the
original coin. If a sender sends their coin to someone, they can send a fraction
of it to the receiver, and then the rest back to themselves, as change.

The \textit{coinfile} is a file that contains the history of a coin, back to
it's minting.


\subsection{The Double Spend Problem}
These coins, on their own, do not work as a currency. The receiver of a coin
does not know whether the sender has already spent that coin. If Alice as a
coin, she can add a transaction giving ownership to Bob on her coinfile, and
send that to Bob, then use the same coin to send one to Tom, and unless Tom and
Bob know each other, there is no way for either of them to know Alice has spent
the coin twice.

\subsubsection{How it has been traditionally solved}
Traditional cryptocurrencies solve this problem by having every coins history be
disclosed publicly on a blockchain. Transactions are only allowed on the
blockchain, if the sender has not already spent the output that is being used as
the input.

\subsubsection{The Issue With This Solution}
The issue, is that hosting a full node can start to take a lot of disk space,
and bandwidth. At the time of writing this, the bitcoin blockchain is 310
gigabytes. It grows at approximately 200 megabytes per day when at around max
capacity. The block size is capped out at around 1.3-1.6 megabytes (although
it's variable due to segwit) in order to prevent the blockchain from getting too
big, too quick. This causes for a bottleneck to occur, where there are more
transactions wanting to confirm than can fit in the next block, or even the next
5, 10, or at the time of writing this, 40 blocks. Because transaction fees are a
market based system, due to supply and demand, the fees spike. At the time of
writing this, a low priority transaction takes approximately 4.5 USD, and in
order to be in the next block, you must pay around 7.5 USD in fees.

One solution that has been tried is merely increasing the block size limit. This
means more transaction can be included in the next block, getting rid of the
bottleneck. This however, leads to centralization issues. Due to the increasing
storage and bandwidth requirements, hosting a full node will become infeasible
for most people, and will only be done by a few.

A better solution is to include as many transactions as possible in the limited
amount of space each block has. In order to do that, you have to decrease the
amount of storage each transaction takes up. Under PADLOCK, each transaction
takes only 48 bytes minimum, compared to the ~190 byte minimum on bitcoin
(including witness data). This results in nearly 4x more throughput when using
the PADLOCK system.

\subsection{How PADLOCK Solves the Double Spend Problem}
PADLOCK solves it by having each transaction of a coin have it's input, output,
and signature each hashed, and then used as an entry on the PADLOCK blockchain.
At a basic level, one can verify whether a coin has already been spent by
hashing the input of the transaction sent to them, and then seeing if that input
has been used on any entries in the blockchain. If it has, you would know that
the coin has already been spent.

Of course, this still doesn't work. Because anyone can make an entry, a bad
actor could make an entry that uses an unspent output that doesn't belong to
them. A solution to this, would be to have the full signature and public key be
part of each entry. However, this would result in transactions taking up a lot
more space (128 bytes if using ed25519).

One optimization would be to remove the public key from the entries, resulting
in a 96 byte single-input, single-output transaction. Someone could make a fake
entry spending an output that doesn't belong to them, however, a receiver of a
transaction could easily verify whether the fake transaction was real, by
checking the signature against the public key of the sender.

To even further the optimization, you can replace the signature with a 16 byte
hash of the signature. Before sending the coinfile with the appended transaction
to the receiver, the sender would first have to scan the blockchain to see if
there are any fake transactions spending their output(s). If there are, the
sender would have to make a signature of each of these, and list them in the new
transaction. The receiver would then be able to hash each of these, and see that
these hashes don't match the ones on chain. They would the know that the coin(s)
being sent to them have not already been spent.

\subsubsection{A Potential Attack to This System, And A Solution to the Attack}
One way to attack this would go as follows:

\begin{enumerate}
    \item Make a transaction that would all of public key A's coins to you.
        Do not worry about the signature for now.
    \item Put an entry on the blockchain for this transaction.
    \item Get A to make a transaction sending you coins.
    \item In the transaction they send you, they would have to include a
        signature of the fake transaction you made.
    \item You can now broadcast the fake transaction with the signature, and
        steal A's coins.
\end{enumerate}

In order to solve this, in order to make a transaction, you would have to make
two signatures. One of the transaction itself, and then one of the entry on the
blockchain (not including the signature hash of course). This means that
despite having a signature for the transaction, the bad actor shown before would
not have a valid signature for the on-chain entry, meaning that if they go to
send their newly acquired coins to someone, the receiver could tell that the
sender made a fake transaction to acquire those coins.

\subsection{Issuance}
Issuance can be done via the 256 byte miner address field in the block header.
Miners can feel free to include whatever they please in this field. Generally,
though they'd want to include some sort of address. A coin could then allow
using the miner address as an input to a transaction.


\section{ASIC Resistance}
The proof of work algorithm used by PADLOCK should be ASIC (application specific
integrated circuit) resistant. ASIC resistance is desirable because ASIC mining
greatly increases the barrier to entry for mining, making it inaccessible for
most people, thus centralizing mining.

In order to achieve ASIC resistance, PADLOCK will cycle through different mining
algorithms.  This makes it uneconomical for miners to invest in ASICs, as those
ASICs will become useless after PADLOCK switches algorithms. However, those
algorithms could be being used by other cryptocurrencies, so an ASIC miner could
simply switch to a different cryptocurrency to mine. This makes it important to
use mining algorithms that are entirely new, or use old ones that are tweaked,
to make them incompatible with ASICs designed for those algorithms.


\section{Privacy}
Privacy is achieved first by pseudonymity. Transactions are associated with
someone's public key, rather than their real name. Of course, the issue there is
that if someone managed to trace a public key back to a real person, that
persons transactions would be publicly known.

A common way to combat that is by sending the change of each transaction to a new
public key. For example, say Alice sends 20 coins to Bob, from her public key, A.
A has 30 coins belonging to it. When she sends the transaction to Bob, instead
of sending the remaining 10 coins back to public key A, she makes a new public
key B, and sends the 10 coins to that. To an outsider, they are unable to tell
which public key was Alice's, and which was Bob's.

PADLOCK further combats this by not having every transaction publicly available.
Instead, they are all in each coinfile. Someone who posses a coinfile would be
able to see many transactions, but it would be incredibly hard to piece together
the entire history of all the transactions. This drastically increases the
privacy.


\section{Anti-Spam Measures}
It would be possible to execute a denial of service attack on PADLOCK by simply
filling the mempool (list of entries that want to be in a block) with
meaningless entries. It would then make it increasingly unlikely that the
legitimate entries would be added to a block. Most cryptocurrencies use a
transaction fee, which makes it very uneconomical for an attacker to create
enough transactions to fill the mempool. Transaction fees are infeasible to
implement under PADLOCK however. It would involve sending many gigabytes
(possibly in the hundreds) worth of coinfiles to the miner, which would take up
immense bandwidth and storage.

Instead, on PADLOCK each transaction must submit a proof of work. The proof of
work algorithm for this will be very similar to hashcash. You must make an md5
hash of the entry with a nonce. The first N bits of the hash must be zeros. This
nonce can be up to 8 bytes, and is then stored with the transaction. The N bits
can be changed. If the mempool is filling up too quickly, nodes can increase the
difficulty.

Instead of being economically expensive to execute a DOS attack, PADLOCK makes
it very computationally expensive.


\section{Conclusion}
PADLOCK is a blockchain system designed to allow for a high throughput, and
privacy friendly cryptocurrency. All the blockchain does is allow for checking
if a coin has already been spent. That is all that is needed for a functioning
digital currency. Because of this, transactions take very little disk space,
allowing a very high throughput. ASIC resistance is achieved through algorithm
switching. Due to transactions not being publicly disclosed on chain, it
achieves a high level of privacy. DOS attacks can be countered by requiring a
proof of work to be submitted with each transaction. All this together creates a very robust base for a digital currency.


\end{document}
